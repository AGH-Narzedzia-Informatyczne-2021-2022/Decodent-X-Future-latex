\section{Patryk Przybysz}
\label{sec:przypatr}
\begin{quote}
\begin{enumerate}
\item Liczba zespolona zapisania w postaci trygonometrycznej:
\[
    z = |z|({\cos\theta} + i{sin\theta})
.\]

\item Wielomian zapisany w postaci ogólnej:
\[
    W(x) = a_k*x^k + a_{k-1}*x*^{k-1} + ... + a_0
.\]

\item Zdjęcie krajobrazu.
\begin{figure}[htbp]
    \centering
    \includegraphics[width=1.2\textwidth]{Pictures/krajobraz.jpg}
    \label{fig:landscape}
\end{figure}
\end{enumerate}
\end{quote}

\begin{itemize}
   \item Algebra liniowa
   \item Geometria
   \begin{itemize}
     \item Geometria przestrzenna
     \item Geometria płaska
     \item Geometria analityczna
   \end{itemize}
   \item Matematyka dyskretna
   \begin{itemize}
     \item Teoria zbiorów
     \item Teoria liczb
     \item Teoria grafów
   \end{itemize}
 \end{itemize}\break

\begin{enumerate}
   \item pomidorowa
   \item grzybowa
   \begin{enumerate}
     \item pieczarkowa
     \item muchumorowa (można zjeść tylko raz)
    \end{enumerate}
    \item barszcz
    \begin{enumerate}
      \item barszcz czerwony
      \item barszcz biały
    \end{enumerate}
 \end{enumerate}

\begin{center}
\textbf{Kombinatoryka} - dział matematyki, zajmujący się badaniem struktur \underline{skończonych} lub \underline{nieskończonych}, ale przeliczalnych.\par 
\textbf{Kombinatoryka} swój rozwój zawdzięcza \textit{rachunkowi prawdopodobieństwa}, \textit{teorii grafów}, \textit{teorii informacji} i innym działom matematyki stosowanej. Stanowi jeden z działów matematyki dyskretnej.\par 
\textbf{Kombinatoryka} posługuje się terminologią niewystępującą w innych działach matematyki, stąd pozorna jej odrębność.
\end{center}

\begin{table}[htbp]
\centering
\begin{tabular}{|c||c||c||c||c||c||c||c||c||c||c|}
x  & 1 & 2 & 3 & 4 & 5 & 6 & 7 & 8 & 9 & 10 \\
1 & 0 & 0 & 4 & 3 & 1 & 4 & 2 & 0 & 0 & 1  \\
2 & 2 & 0 & 1 & 1 & 3 & 5 & 3 & 2 & 3 & 0  \\
3 & 0 & 0 & 0 & 2 & 2 & 6 & 2 & 2 & 1 & 0  \\
4 & 0 & 9 & 0 & 0 & 1 & 7 & 1 & 9 & 1 & 0  \\
5 & 1 & 8 & 6 & 2 & 0 & 8 & 1 & 5 & 8 & 2 \\
6 & 1 & 8 & 6 & 2 & 0 & 0 & 5 & 6 & 7 & 3  \\
7 & 1 & 8 & 6 & 2 & 0 & 2 & 0 & 3 & 2 & 1  \\
8 & 1 & 8 & 6 & 2 & 0 & 1 & 3 & 0 & 3 & 4  \\
9 & 1 & 8 & 6 & 2 & 0 & 1 & 2 & 1 & 0 & 1  \\
10 & 1 & 8 & 6 & 2 & 0 & 0 & 0 & 4 & 6 & 0 \\
\end{tabular}
\label{tab:adj_matrix}
\caption{Macierz sasiedztwa skierowanego grafu ważonego}
\end{table}
\newpage
