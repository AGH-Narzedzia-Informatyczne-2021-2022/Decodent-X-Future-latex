\newpage
\section{Maciej Pieniążek}

\subsection{Oto jest duży nagłówek}
Tutaj może zobaczyć jakąś tam ścianę tekstu, a tutaj możemy zobaczyć parę bajerów \% \textbf{jestem debeściak}. Wracamy do wielkiej ściany tekstu żeby pokazać jak mogą wyglądać takie właśnie teksty. Przechdzimy do list:
\begin{enumerate}
    \item Pierwszy element listy
        \begin{itemize}
            \item podlista1
            \item fajne narzędzie
            \item do wypunktowywania
            \item rzeczy
        \end{itemize}
    \item Drugi element listy
        \begin{itemize}
            \item podlista2
            \item fajne narzędzie
            \item do wypunktowywania
            \item rzeczy
        \end{itemize}
\end{enumerate}

\setlength{\arrayrulewidth}{0.5mm}
\setlength{\tabcolsep}{18pt}
\renewcommand{\arraystretch}{1.5}
\begin{tabular}{ |p{3cm}|p{3cm}|p{3cm}|  }
\hline
\multicolumn{3}{|c|}{Afrykańskie kraje} \\
\hline
Kraj& Jakiś tam parametr&inny parametr \\
\hline
Afganistan& 12543 &werwf \\
Chad& A345X35   & dfgs \\
RPA&2345& sdfgh \\
Nigeria&61346& sdfh \\
Etiopia& 243&  sdf \\
Kenia& 2324 & dgf   \\
Zimbabwe& 3245& AbfgGytO \\
\hline
\end{tabular}

\subsection{Drugi nagłówek}
\underline{DOKŁADNIE TAK} \\ 
i oto są też różne \emph{bajery}. dalej wracamy do potężnej ściany tekstu i budujemy ścianę z wyrazów. Na koniec jeszcze równanie: \\
\begin{math}
e=mc^2 
\end{math}
nie no dobra coś trudniejszego
\begin{math}
f(\xi, \eta)=\left[\begin{array}{l}
f^{1}(\xi, \eta)\\
f^{2}(\xi, \eta)\\
f^{3}(\xi, \eta)\\
\end{array}\right]=\left[\begin{array}{l}
\xi+\eta\\
\xi^{2}+\eta^{2}\\
\xi^{3}+\eta^{3}\\
\end{array}\right]
\end{math}

\item Zdjecie oblezenia konstantynopolu
\begin{figure}[htbp]
    \centering
    \includegraphics[width=1.2\textwidth]{Pictures/obrazek.jpeg}
    \label{fig:oblezenie konstantynopolu}
\end{figure}
\newpage